%******************************************************************************
% KOMA-Script article scrartcl
%******************************************************************************
\documentclass[10pt, b5paper]{article} 

% Add here all the packages that will be used in the document
%******************************************************************************
% Packages 
%******************************************************************************
% For hyperlinks 
\usepackage{url}

% Add no chapters to the article 
\usepackage[nochapters]{classicthesis}

% Setup the page geometry
\usepackage{geometry}
\geometry{a4paper, total={210mm, 297mm}, 
left=20mm, right=20mm, top=20mm, bottom=20mm}

% Use tikz to add watermarks to the document
\usepackage{blindtext,tikz}
\usetikzlibrary{calc}

% Use the classicthesis style for the style of the document
\usepackage[nochapters]{classicthesis} 

% Use the currvita style for the layout of the document
\usepackage[LabelsAligned]{currvita} 

% Required for adding links	and customizing them
\usepackage{hyperref} 

 % Set link colors
\hypersetup{colorlinks, breaklinks, urlcolor=Maroon, linkcolor=Maroon}


% Add a list of new commands 
%******************************************************************************
% Commands 
%******************************************************************************
\newcommand{\latex}{\LaTeX\xspace}
\newcommand{\tex}{\TeX\xspace}

\usepackage{listings}
\usepackage{color}

\definecolor{dkgreen}{rgb}{0,0.6,0}
\definecolor{gray}{rgb}{0.5,0.5,0.5}
\definecolor{mauve}{rgb}{0.58,0,0.82}

\lstset{frame=tb,
  language=Java,
  aboveskip=3mm,
  belowskip=3mm,
  showstringspaces=false,
  columns=flexible,
  basicstyle={\small\ttfamily},
  numbers=none,
  numberstyle=\tiny\color{gray},
  keywordstyle=\color{blue},
  commentstyle=\color{dkgreen},
  stringstyle=\color{mauve},
  breaklines=true,
  breakatwhitespace=true,
  tabsize=3
}

%******************************************************************************
% DOCUMENT STARTS HERE 
%******************************************************************************
\begin{document}

% TITLE 
\title{\rmfamily\normalfont\spacedallcaps{
Report \#{1}
}}

% PROJECT NAME -> ADD YOUR PROJECT NAME HERE
\author{{\small Automatic Mandible Segmentation Using VTK}}

% AUTOMATIC DATE -> DON'T CHANGE MANUALLY
\date{\footnotesize{\today}}

% MAKE THE TITLE -> DON'T CHANGE MANUALLY
\maketitle

% DON'T INCLUDE THE ABSTRACT FOR THE MOMENT
% % \begin{abstract}
% \noindent Abstract
% \end{abstract}
 
%******************************************************************************
% TABLE OF CONTENTS (uncomment to show / comment to hide)
%******************************************************************************     
% \tableofcontents


%******************************************************************************
% Report Content
%******************************************************************************
\section{Report Details}
\begin{center}
\begin{tabular}{ l | c }
\hline 
Report ID & 1  \\ % Change the sprint ID here 
\hline 
Report Duration & 1 Week \\ % Change the duration here 
\hline 
Beginning & 17.10.2016 \\ % Change the start data here
\hline 
End & 24.10.20.16 \\ % Change the end data here
\hline 
\end{tabular}
\end{center}

%\section{Objectives}
\section{Original Objectives}
\begin{enumerate}
\item Rescalling scalar Values to pick up negative scalars.
\item Accessing scalar value of a Pixel from its Coordinates (i, j, k).
\item Thresholding to segment bone structures.
\item Connect VTK with Qt for GUI Design.
\item Reading Master Thesis and understanding the Algorithm.
\item Implement lableling and connectivity algorithm for segmentation.
\end{enumerate}

\section{Accomplished Objectives}
\subsection{Rescalling scalar Values to pick up negative scalars}
\textit{GetPixelRepresentation()} method of \textit{vtkDICOMImageReader} tells us whether stored values signed or unsigned, A zero is a unsigned quantity. A one indicates a signed quantity.
To rescale data we must have the rescale slope and rescale offset As the Equation is 
\begin{equation}
final Value = Slope* initial Value + Offset 
\end{equation}
and We can obtain them from \textit{GetRescaleOffset()} and \textit{GetRescaleSlope()} methods of \textit{vtkDICOMImageReader}. 
We Also have to know the number of bits for each pixel and data type used. So use \textit{GetBitsAllocated()} and  \textit{GetDataScalarType()} methods of \textit{vtkDICOMImageReader}, Check vtkType.h (Note : VTK\_SHORT = 4).

Now use \textit{vtkImageShiftScale} to rescale the values  as follow :

\begin{lstlisting}
vtkImageShiftScale * shifter = vtkImageShiftScale::New();

//set the scalar slope and offset     
//Note offset obtained by GetRescaleOffset() must be reversed      
shifter->SetScale(rescaleSlope);
shifter->SetShift(-1*rescaleOffset);
    
//Set data type to store  pixel values
shifter->SetOutputScalarTypeToShort();
       
//Get Data From Dicom Reader     
shifter->SetInputConnection(reader->GetOutputPort());
shifter->Update();

//Now Get Image Data after re-scaling    
vtkImageData* rescaledImageData = shifter->GetOutput();

\end{lstlisting}

\subsection{Accessing scalar value of a Pixel from its Coordinates (i, j, k)}
\textit{GetScalePointer(double point[3])} method of \textit{vtkImageData} object returns a pointer to the scalar value at specific i, j, and k stored in point[3] array, So we can access the scalar value of each point in the 3D volume.

\subsection{Thresholding to segment bone structures}
The Next Step is to threshold the data to emphasis only on the bone structure, 1200 is a good threshold after rescaling the data that will cull out the skin and soft tissues, vtk has a \textit{vtkThresholdPoints} class and we can
set the upper and lower threshold of the data and apply thresholding in different manners like \textit{ThresholdByUpper()} 
or \textit{ThresholdByLower()}, and so on.

But the problem of this class is that i can not access data points as in \textit{vtkImageData} class So I think the best way 
is to do it manually using an if condition to get values upper certain threshold.

\subsection{Connect VTK with Qt for GUI Design}
I think we will need to design a Graphical User Interface to enable the user to load different volumes and initialize the volume of interest  and to do some other functions, unfortunately VTK was built without enabling Qt ( C++ GUI Library). So I have to rebuild the VTK project and enable Qt module and successfally running a Simple GUI example with VTK. 

\subsection{Reading Master Thesis and understanding the Algorithm}
I have read the mathods cahpter but I was confused, What is the function of separation line in all slides?, Why to check the connectivity of only 13 points of neighbours?, what is the omputational complexity of these algorithm and can we speed it up?. 
\section{Missed Objectives}
\subsection{Understanding the Segmentation Algorithm}
It was difficult for me to understand the total algorithm.
\subsection{Implement lableling and connectivity algorithm for segmentation}
I have no enough time to implement the algorithm, and It is also not clear for me.
\section{Other Ideas}
\subsection{ITK}
I have found another library (ITK) for segmentation and registration, I have only installed it but I don't know how to use it and how to connect it with VTK, I need further research to know if this library will be useful for us or not.

 
%\section{Next Step}
%Summarize the main points that will be taken into consideration in the next sprint

%\section{Next Sprint Objectives}
%\begin{enumerate}
%\item Objective 1
%\item Objective 2
%\item Objective 2
%\end{enumerate}


    
   
    
%******************************************************************************
% DOCUMENT ENDS HERE 
%******************************************************************************
\end{document}